\documentclass{article}

\usepackage[backref=page]{hyperref}


\title{Why all programmers want continuations\\\vspace{2mm}{\large (but use callbacks instead)}}

\author{Gabriel Kerneis}

\date{12 April 2015}

\begin{document}

\pagestyle{plain}

\pagenumbering{arabic}

\maketitle

\paragraph{Abstract:}
%
Have you ever wondered why callbacks are so pervasive in modern programming
languages, and yet so hated that there is such an idiom as ``callback hell''?
Have you ever scratched your head so hard that you started losing your hair
while debugging a maze of twisty little functions all alike? Have you ever
wished you could write straightforward, linear, synchronous code, and let your
programming language handle concurrency? This is 2015: why isn't your compiler
able to link stack frames by itself as soon as you are writing asynchronous
code? As it turns out, your compiler can in fact do this for you, and much
more. It just needs a gentle push in the right direction.

This talk is a tutorial on escaping callback hell with promises and generators;
examples are in Javascript, but should be accessible to any interested
programmer.  We first build a minimal promise implementation from first
principles, discovering how the underlying hidden monad makes
continuation-passing style programming easier and safer.  Then, we go one step
further, and throw generators into the mix to recover a direct, coroutine
style, restoring sanity and reaching true enlightenment. We conclude with a
brief tour of other popular programming languages, and discover that the essential
building blocks are already available in most cases. Educating users about them
is left as an exercise to the reader.

\paragraph{Acknowledgments:}
%
The author is grateful to Matt Greer and Jake Archibald for their tutorials on
promises, heavily reused in this presentation. He also wishes to thank the many
callback lovers (and the occasional continuation haters!) he has pitched this
talk to in the last few months. Their insightful, if often fairly defensive,
feedback has been the main motivation for giving this talk.


{\small
\paragraph{Disclaimer:}
%
The opinions expressed in this tutorial are those of the author, and do not
necessarily reflect the official position of his employer.  No callback has
been harmed during the preparation of this tutorial.}


\end{document}
